\documentclass[a4paper]{letter}
\usepackage[a4paper, margin=2.5cm]{geometry}
\usepackage{url}

\begin{document}

\begin{letter}{}
\vspace*{-5.0cm}

\opening{Dear Editor:}


 The enclosed manuscript, ``Prosit Transformer: A transformer for prediction of MS2 spectrum intensities'', discusses the benefit of a promissing deep learning technique, Transformers, for Mass Spectrometry-based Proteomics. We demonstrate a successfull emplotment of the technology by retraining the Prosit model for predicting MS2 spectrum intensities. We call this new predictor Prosit Transformer, to distiguish it from its predecessors. We believe that this predictor is just a first of many Transformer-based predictors that will transform the future of analysis of data from mass spectrometers.
 
The manuscript is an original contribution and none of the work it describes has been published elsewhere, but we would like to keep the right to make it available through bioRxiv if needed. The manuscript will remain with you and will not be submitted elsewhere until you made a decision as to its suitability
for publication in Journal of Proteome Research.

We suggest the following six experts in the analysis of proteomics data as
potential reviewers:

\begin{itemize}

\item Professor Lieven Clement,
Ghent University, \\
\url{Lieven.Clement@ugent.be}

\item Dr. Timo Sachsenberg,
University of T\"{u}bingen,\\
\url{sachsenb@informatik.uni-tuebingen.de}

\item Professor Lennart Martens,
Gent University, \\
\url{lennart.martens@ugent.be}


\item Dr. Magnus Palmblad,
Leids Universitair Medisch Centrum\\
\url{n.m.palmblad@lumc.nl}

%\item Professor Ruedi Aebersold, Swiss Federal Institute of Technology (ETH), \\
%\url{aebersold@imsb.biol.ethz.ch}


\item Dr. Samuel H. Payne, Pacific Northwest National Laboratory, \\
\url{Samuel.Payne@pnnl.gov}

\end{itemize}

\vspace*{1.5em}

Sincerely,\\[2em]
Mr. Markus Ekvall and\\
Dr. Lukas K\"all, Science~for~Life~Laboratory, School of
Engineering Sciences in Chemistry, Biotechnology and Health,\\
Royal Institute of Technology - KTH, 17165 Solna, Sweden\\
\url{lukas.kall@scilifelab.se}\\
Tel: +46 8 52481196, Fax: +46~8~52481425

\end{letter}
\end{document}
